\documentclass{amsart}
\usepackage{amsmath}
\usepackage{array}
\usepackage{tabu}
\usepackage{mathtools}
\pagestyle{plain}
\everymath{\displaystyle} 

\DeclarePairedDelimiter\ceil{\lceil}{\rceil}
\DeclarePairedDelimiter\floor{\lfloor}{\rfloor}
\providecommand{\bfloor}[1]{\biggl \lfloor #1 \biggr \rfloor }
\providecommand{\bceil}[1]{\Bigl \lceil #1 \Bigr \rceil }
\DeclarePairedDelimiter\abs{\lvert}{\rvert}
\DeclareMathOperator{\sign}{sign}

\newcommand{\Mod}[1]{\ \text{mod}\ #1}
\newcommand{\R}{\mathbb{R}}
\newcommand{\B}{\mathbb{B}}
\newcommand{\N}{\mathbb{N}}
\newcommand{\Z}{\mathbb{Z}}
\newcommand{\x}{\times}
\newcommand{\m}{\rightarrow}

% -------------- TODO --------------
% -	Check _ALL_ equations for validity and different base behavior
% -	Finish
%	-	Descriptions for _ALL_ branches.
% 	-	First Branch
%	-	Prime checker, Factorization
% -	Add
%	-	Boolean Algebra 'pre Foundation'
% 	-	Draw
%	-	Number digits for floats.
% - Attempt to fix 'Missing }' error

\title{Math Functions}

\author{Sriram G. Bhat, Enrico Z. Borba, \& Michael K. Opara}

\begin{document}

	\begin{abstract}
		These functions were discovered as they were needed, to feed the satisfaction of curious individuals with an interest in mathematics and computer science. The first function of this  nature to be discovered was $min(x, y)$ and was sufficient to create an interest in discovering functions of a similar kind. As more fundamental functions were discovered, such as $eq(x, y)$, more complex functions could be developed. Functions such as Look and Say's $L(x)$ and $sort(x)$ were quickly described and are proof of the powerful and useful nature of these fundamental functions.
	\end{abstract}
	\maketitle

	\pagebreak

	\tabulinesep=1.2mm
	
	\section*{0. Boolean Operations}

		\noindent\makebox[\textwidth]{
			\begin{tabu} 
			{| >{\centering\arraybackslash}m{2.5cm} | >{\centering\arraybackslash}m{2cm} | >{\centering\arraybackslash}m{8cm} | m{4.6cm} |}
			
			\hline

			Boolean Set			& $\B$				& $\left\{0, 1\right\}$					& The set of all possible boolean values
			\\\hline

			AND 				& $and(x, y):
								\B\x\B\m\B$			& $x \cdot y$							& Returns 1 if $x = y = 1$, 0 otherwise. 
			\\\hline

			OR 					& $or(x, y):
								\B\x\B\m\B$			& $\bceil{\frac{x + y}{2}}$				& Returns 1 if $x = 1, y = 1$ or both. 0 otherwise.								
			\\\hline

			NOT 				& \shortstack{
								$not(x):$ \\  
								$\B\m\B$}			& $1 - x$								& Returns 0 if $x = 1$, 1 otherwise.
			\\\hline
			\end{tabu}
		}
	\pagebreak

	\section{Foundation}

		\noindent\makebox[\textwidth]{
			\begin{tabu} 
			{| >{\centering\arraybackslash}m{2.5cm} | >{\centering\arraybackslash}m{2cm} | >{\centering\arraybackslash}m{8cm} | m{4.6cm} |}
			
			\hline

			Floor$^*$ 				& $\floor{x}:
									\R\m\Z$		 		& $x - x \Mod 1$									& Definition of floor.
 			\\\hline

 			Ceiling$^*$ 			& $\ceil{x}:
 									\R\m\Z$				& $x + (-x \Mod 1)$									& Definition of ceiling.
 			\\\hline

			Minimum  				& $min(x, y):
									\R\x\R\m\R$ 		& $\frac{x + y}{2} - \frac{|x - y|}{2}$ 			& Returns the lesser of $x$ and $y$.
			\\\hline
			
			Maximum  				& $max(x, y):
									\R\x\R\m\R$ 		& $\frac{x + y}{2} + \frac{|x - y|}{2}$ 			& Returns the larger of $x$ and $y$.
			\\\hline

			Equality 				& $eq(x, y): 
									\R\x\R\m\B$ 		& $2^{\ceil*{\abs{x - y}}} \Mod 2$ 					& Returns 1 if $x$ and $y$ are equal, 0 otherwise.
			\\\hline

			Signum$^*$				& $sign(x):
									\R\m\Z$				& $\floor*{\frac{x}{\abs{x} + 1}} 
														+ \ceil*{\frac{x}{\abs{x} + 1}}$ 					& Returns the sign of $x$. -1 if it's negative, 1 if it's 
																											positive, 0 otherwise. Is always defined.
			\\\hline

			Digit At 				& $dat(x, b, i): 
									\R\x\N\x\Z\m\N$ 	& $\bfloor{\frac{\abs{x}}{b^i}} \Mod b$ 			& Returns the digit at index $i$ of number $x$ in base $b$. 
			\\\hline

			Number of Digits 		& $nd(x, b):
									\N\x\N\m\N$ 		& $\bceil{\log_{b}{(x + 1)}}$ 						& Returns the number of digits of $x$ in base $b$.
			\\\hline

			\end{tabu}
		}
	\pagebreak

	\section{First Branch}
		\noindent\makebox[\textwidth]{
			\begin{tabu}
			{|>{\centering\arraybackslash}m{2.5cm}|>{\centering\arraybackslash}m{2cm}|>{\centering\arraybackslash}m{8cm}| m{4.6cm} |}
			
			\hline

			Equality 					& $eq(x, y): 
										\R\x\R\m\B$ 		& $1 - \ceil*{ \frac{\abs{x - y}}{\abs{x - y} + 1}}$ 			& Returns 1 if $x$ and $y$ are equal, 0 otherwise.	
			\\\hline

			Equality 					& $eq(x, y): 
										\R\x\R\m\B$ 		& $1 -\abs{sign(x - y)}$ 										& Returns 1 if $x$ and $y$ are equal, 0 otherwise.
			\\\hline

			Compare						& $com(x, y):
										\R\x\R\m\Z$			& $sign(x - y)$													& Returns 1 if $x > y$, 0 if $x = y$, and -1 if $x < y$.
			\\\hline

			Less Than					& $lt(x, y):
										\R\x\R\m\B$			& $eq(-1, com(x, y))$										& Returns 1 if $x < y$, 0 otherwise.
			\\\hline

			Greater Than				& $gt(x, y):
										\R\x\R\m\B$			& $eq(1, com(x, y))$										& Returns 1 if $x > y$, 0 otherwise.
			\\\hline

			Less Than or Equal To		& $le(x, y):
										\R\x\R\m\B$			& $eq(min(x, y), x)$										& Returns 1 if $x \leq y$, 0 otherwise.
			\\\hline

			Greater Than or Equal To	& $ge(x, y):
										\R\x\R\m\B$ 		& $eq(max(x, y), x)$										& Returns 1 if $x \geq y$, 0 otherwise.
			\\\hline

			Sum Digits					& $sd(x, b):
										\N\x\N\m\N$			& $\sum_{i = 0}^{nd(x, b) - 1}{dat(x, b, i)}$				& Returns the sum of the digits in $x$ in base $b$.
			\\\hline

			To Base						& $tb(x, b):
										\N\x\N\m\N$			& $\sum_{i = 0}^{nd(x, b) - 1}{dat(x, b, i) \cdot 10^i}$	& Returns a base 10 representation of the base $b$ representation of 
																														$x$. 
			\\\hline

			Signum   					& $sign(x):
										\R\m\Z$ 			& $\frac{\abs{x}}{x + eq(x, 0)}$ 							& Returns the sign of $x$. -1 if it's negative, 1 if it's positive, 0 
																														otherwise. Is always defined.
			\\\hline

			Reverse						& $rev(x, b):
										\N\x\N\m\N$			& $\sum_{i = 0}^{nd(x, b) - 1}{dat(x, b, i) \cdot 
															10^{nd(x, b) - i - 1}}$										& Returns the reverse of the number $x$ in base $b$.
			\\\hline

			Whole Number 			& \shortstack{
									$w(x):$
									\\$\R\m\B$}				& $\floor{\cos(\pi x)^2}$									& Returns 1 if $x$ is whole, $x\in\Z$, 0 otherwise.
			\\\hline

			\end{tabu}
		}
	\pagebreak

	\section{Look and Say}
		\noindent\makebox[\textwidth]{
			\begin{tabu}
			{|>{\centering\arraybackslash}m{2.5cm}|>{\centering\arraybackslash}m{2cm}|>{\centering\arraybackslash}m{8cm}| m{4.6cm} |}
			
			\hline
			Look and Say Counter		& $C_\lambda(x, i):
										\N\x\N\m\N$			& $sd(x \Mod 10^i, 10)$											& Returns the sum of the digits of $x$ given a maximum left 
																															index $i$.
			\\\hline

			Unpadded Difference			& $\delta(x):
										\N\m\N$				& $\sum_{i = 0}^{nd(x, 10) - 2}{10^i \cdot 
																	(1 - eq(dat(x, 10, i), dat(x, 10, i + 1)))}$ 			& Returns the differences of digits of $x$. 
			\\\hline

			Padded Difference 			& \shortstack{
										$\Delta(x):$ \\
										$\N\m\N$}			& $10^{nd(x, 10)} + 10 \cdot \delta(x) + 1$ 					& Returns the padded digit differences of $x$.
			\\\hline

			Leftmost Index 				& $il(x, i):
										\N\x\Z\m\Z$			& $nd(x \Mod 10^{i + 1}, 10) - 1$								& Returns the leftmost index of a non-zero digit within $x$,
																															given a left starting index of $i$.
			\\\hline

			Rightmost Index 			& $ir(x, i):
										\N\x\Z\m\Z$			& $nd(x, 10) - nd\Bigg(rev\bigg(\bfloor{
																	\frac{x}{10^{i + 1}}}, 10\bigg), 10\Bigg)$ 				& Returns the rightmost index of a non-zero digit within $x$, given a 
																															right starting index of $i$.
			\\\hline

			Look and Say				& \shortstack{
										$L(x):$ \\ 
										$\N\m\N$}			& $\begin{aligned} \sum_{i = 0}^
																{nd\big(\Delta(x), 10\big) - 2}{ 
															&\Bigg(dat\big(\Delta(x), 10, i\big) 
																\cdot \bigg(\Big(ir\big(\Delta(x), i\big) - i\Big) 		\\
															&\cdot 10^{1 + 2 
																\cdot C_\lambda\big(\Delta(x), i\big)} + dat(x, b, i) 	\\
															&\cdot 10^{2\cdot C_\lambda\big(\Delta(x), i\big)}
																\bigg)\Bigg)} 
															\end{aligned}$						
																															& Returns the following term in John Conway's Look and Say sequence 
																															given a term $x$.
			\\\hline

			\end{tabu}
		}
	\pagebreak

	\section{Sorting}
		\noindent\makebox[\textwidth]{
			\begin{tabu}
			{|>{\centering\arraybackslash}m{2.5cm}|>{\centering\arraybackslash}m{2cm}|>{\centering\arraybackslash}m{8cm}| m{4.6cm} |}

			\hline

			Counter			& $C_\sigma(x, n, j)$	& $\sum_{i = 0}^{nd(x, 10) - 1}{eq(-1, com(n, dat(x, 10, i)))} 
													+ \sum_{i = 0}^{j - 1}{eq(n, dat(x, 10, i))}$ 							& Returns the number of digits less than $n$ within $x$ and
																															the number of occurences of digit $n$ with an index less 
																															than $j$. Used to determine the index of the current digit
																															when sorted.
			\\\hline

			Sort			& $sort(x)$				& $\begin{aligned}\sum_{i = 0}^{nd(x, 10) - 1}{
													&\Big(dat(x, 10, i) \\
													&\cdot 10^{nd(x, 10) - 1 - C_\sigma(x, dat(x, 10, i), i)
														}\Big)}
													\end{aligned}$															& Returns the sorted digits of $x$.
			\\\hline

			Counter 
			(Vector)		& $C_\sigma(A, n, j)$	& $\sum_{i=1}^{n}{lt(A_j, A_i)} + \sum_{i=1}^{j}{eq(A_j, A_i)}$			& Returns the index of the element at index $j$ in vector
																															$A$ of size $n$ when in its respective sorted vector.																	
			\\\hline

			Sort 			& $A^S$					& $[A_{C_\sigma(A, n, i)}](i=1, ..., n)$								& Returns the sorted representation of vector $A$ of size
																															$n$, in decreasing order.
			\\\hline
			\end{tabu}
		}
	\pagebreak

	\section{Primality}
		\noindent\makebox[\textwidth]{
			\begin{tabu}
			{|>{\centering\arraybackslash}m{2.5cm}|>{\centering\arraybackslash}m{2cm}|>{\centering\arraybackslash}m{8cm}| m{4.6cm} |}

			\hline

			Is Prime 				& $p(x)$		& $1 - eq(1, x) - eq\Bigg(0, \prod_{i = 2}^{\sqrt{x}}{(x \Mod i)}\Bigg)$	& Returns 1 if $x$ is prime, 0 otherwise.
			\\\hline

			Multiplicity			& $m(x, f)$		& $ir(tb(x, f), - 1)$														& Returns the multiplicity of the factor $f$ of $x$.
			\\\hline

			Prime Factorization		& $P(x)$		& $\sum_{i = 2}^{x}{p(i) \cdot m(x, i) \cdot 10^i}$							& Returns the prime factorization of $x$ in the form 
																																of a base 10 number with its digits' indices as the
																																factors and the value of the digits as their
																																multiplicities.	
			\\\hline
			\end{tabu}
		}
	\pagebreak
	
	\section{Rendering}
		\noindent\makebox[\textwidth]{
			\begin{tabu}
			{|>{\centering\arraybackslash}m{2.5cm}|>{\centering\arraybackslash}m{2cm}|>{\centering\arraybackslash}m{8cm}| m{4.6cm} |}

			\hline

			Render 					& $ren(x, y, d, h)$		& $dat\Bigg(\floor*{\frac{y + d \cdot h}{h}}, 
																2, x \cdot h + y \Mod h\Bigg) > 0$								& Renders the monochromatic image encoded in
																																$d$ of height $h$ at cornered at $(0, 0)$ when
																																plotted against the $xy$-plane.
			\\\hline
			\end{tabu}
		}
	\pagebreak

	\section{Analysis}
		\noindent\makebox[\textwidth]{
			\begin{tabu}
			{|>{\centering\arraybackslash}m{2.5cm}|>{\centering\arraybackslash}m{2cm}|>{\centering\arraybackslash}m{8cm}| m{4.6cm} |}

			\hline
			Dirac Delta  			& $\delta_0(x)$			& $\lim_{a \to 0} \frac{1}{a\sqrt{\pi}}e^{-x^2/a^2}$ 				& The Dirac Delta function is special in that it
																																is 0 everywhere except at $x = 0$, where it is $\infty$.
																																Its integral over the real line is equal to 1.
			\\\hline

			Dirac Delta Comb
			of Period Zero 			& $\textrm{III}_0(x)$	& $\lim_{T \to 0} \sum_{k = -\infty}^{\infty}\delta_0(x - kT)$		& The Dirac Delta comb with a period of zero can be thought
																																of as a function similar to the Dirac Delta function with its
																																spike at all real values of $x$. If certain points of the comb can be isolated, then the integral of said isolation can result in
																																useful integer values.
			\\\hline

			Unapply Size			& $u(f, y_0)$			& $\int_{-\infty}^{\infty}eq(f(x), y_0) 
															\cdot \textrm{III}_0(x)dx$ 											& Returns the number real $x$-values such that $f(x) = y_0$.
			\\\hline	

			\end{tabu}
		}
	\pagebreak

	\section{Sets}
		\noindent\makebox[\textwidth]{
			\begin{tabu}
			{|>{\centering\arraybackslash}m{2.5cm}|>{\centering\arraybackslash}m{2cm}|>{\centering\arraybackslash}m{8cm}| m{4.6cm} |}

			\hline

			Prime Numbers			& $\mathbb{P}$			& $\left\{x \mid x \in \N \land p(x) = 1\right\}$					& The set of all prime numbers		
			\\\hline

			Prime Factors 			& $\mathbb{P}_a$		& $\left\{x \mid x, a \in \N \land 
															2 \leq x \leq a \land p(x)\cdot m(a, x) > 0 \right\}$				& The set of the prime factors
																																of $a$.
			\\\hline	

			\end{tabu}
		}
	\pagebreak

	\section{The Ruler Function}
		\noindent\makebox[\textwidth]{
			\begin{tabu}
			{|>{\centering\arraybackslash}m{2.5cm}|>{\centering\arraybackslash}m{2cm}|>{\centering\arraybackslash}m{8cm}| m{4.6cm} |}

			\hline

			The Ruler Function		& $r(x)$			& $\sum_{i=0}^{\ceil*{\log_2(x + 1)}}
														{i\cdot eq(x\Mod 2^{i+1}, 2^i-1)}$ 			              	& Returns the value of the ruler function 
																													corresponding to index $x$; the minum value is $0$. 		
			\\\hline

			\end{tabu}
		}
	\pagebreak
\end{document}