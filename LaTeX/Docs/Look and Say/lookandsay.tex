\documentclass{article}
\usepackage{amsmath}

\everymath{\displaystyle}

\begin{document}
	\title{The Look-and-Say Function}
	\author{Enrico Borba}
	\date{May 2015}
	\maketitle

	\begin{abstract}
		The Look-and-Say sequence was originally introduced and analyzed by John Conway. 
		The sequence begins as follows:
		\begin{center}
		$1, 11, 21, 1211, 111221, 312211, 13112221, 1113213211...$
		\end{center}
		To generate a member of the sequence from the previous member, read 
		the digits of the previous member, counting the number of digits in 
		groups of the same digit. For example:
		\begin{enumerate}
			\item 1 is read off as ``one 1'' or 11.
			\item 11 is read off as ``two 1s'' or 21.
			\item 21 is read off as ``one 2, then one 1'' or 1211.
			\item 1211 is read off as ``one 1, then one 2, then two 1s'' or 111221.
		\end{enumerate}
		This sequence has been extensively analyzed in terms of its properties, but no formal 
		definition for a function which generates its terms has been developed.
		Until now. Let us venture to where no man has gone before...
	\end{abstract}
	\clearpage
	
	\section{Introduction}
	Before attempting to create a function which can generate the terms of the Look-and-Say
	sequence, a more rigorous algorithmic method of arriving at the terms should be described.
	
	\subsection{Differences}
		Some points of interest regarding the generation of a member of the sequence are the
		differences found between the digits of the previous member which was used to generate it.
		``Quantifying'' these differences would consist of creating a number whose digits could 
		be used to represent a member of the sequence. For example, embedding ``comparisons'' in 
		between each of the digits of a member such as $L_0 = 111221$ may look something like this:
		\begin{center}
		${1}_0{1}_0{1}_1{2}_0{2}_1{1}$
		\end{center}
		Here, a $0$ represents an ``equality'' in the two digits it resides between,
		while a $1$ designates a ``difference''. Consequently, the number generated using this 
		technique, called $\Delta_0$, would be
		\begin{center}
		$00101$\\
		or\\
		$101$
		\end{center}
		Now, it is important to preserve the leading $0$s as they have significant value in 
		regards to the current member being used to generate $\Delta_0$. A simple way to preserve
		these leading zeros would be to ``pad'' the resulting $\Delta_0$ with $1$s as such:
		\begin{center}
		$_1{1}_0{1}_0{1}_1{2}_0{2}_1{1}_1$
		\end{center}
		\begin{center}
		$\Delta_0 = 1001011$
		\end{center}
		With $\Delta_0$, the differences between the digits of $L_0$ can be identified and used to 
		generate the next member.

		\subsection{Counting}
		In between each instance of a $1$ in $\Delta_0$ are digits of the same kind within $L_0$. 
		Put simply, the instances of $1$s in $\Delta_0$ can be used to separate $L_0$ into its 
		``groups'' of digits. 
		\begin{center}
		$_1{1}_0{1}_0{1}_1{2}_0{2}_1{1}_1$\\
		$_1{1}{1}{1}_1{2}{2}_1{1}_1$
		\end{center}

		\clearpage

		\noindent For example, here are $L_0$ and $\Delta_0$ with their digits' indices labeled:
		\begin{center}
		$L_0 = 
		\underset{5}{1}\underset{4}{1}\underset{3}{1}\underset{2}{2}\underset{1}{2}\underset{0}{1}
		$
		and
		$\Delta_0 = 
		\underset{6}{1}\underset{5}{0}\underset{4}{0}\underset{3}{1}\underset{2}{0}\underset{1}{1}
		\underset{0}{1}
		$
		\end{center}
		The first instance of a $1$ in $\Delta_0$ (starting from the right) occurs at index $0$.
		\begin{center}
		$\Delta_0 = 100101\underset{0}{1}$
		\end{center}
		The second instance of a $1$ occurs at index $1$.
		\begin{center}
		$\Delta_0 = 10010\underset{1}{1}1$
		\end{center}
		The difference in these indices is $1$. This indicates that these two $1$s surround a 
		``group'' of digits only one digit long. In this case, these first two instances of $1$s 
		``surround'' a $1$ in $L_0$ (placed in brackets for clarity).
		\begin{center}
		$_1{1}{1}{1}_1{2}{2}[_1{1}_1]$
		\end{center}
		The next pair of $1$s in $\Delta_0$ occur in indices $3$ and $1$.
		\begin{center}
		$\Delta_0 = 100\underset{3}{1}0\underset{1}{1}1$
		\end{center}
		This pair of $1$s correspond to the second group in $L_0$. These two indices have a 
		difference of $2$ and therefore ``surround'' a ``group'' of digits in $L_0$ that is two 
		digits long.
		\begin{center}
		$_1{1}{1}{1}[_1{2}{2}_1]{1}_1$
		\end{center}
		Carrying this forward to the final group of digits in $L_0$, the difference in indices is
		$3$
		\begin{center}
		$\Delta_0 = \underset{6}{1}00\underset{3}{1}011$ 
		\end{center}
		Since this is the third pair of $1$s in $\Delta_0$, it corresponds to the third 
		``group'' in $L_0$, which is composed of $1$s.
		\begin{center}
		$[_1{1}{1}{1}_1]{2}{2}_1{1}_1$
		\end{center}
		Appending each of these counts and digits in $L_0$ together, from right to left, generates 
		the next member, $L_1$
		\[
			_1{1}{1}{1}_1{2}{2}[_1{1}_1] \tag*{one one}
		\]
		\[
			_1{1}{1}{1}[_1{2}{2}_1]{1}_1 \tag*{two two}
		\]
		\[
			[_1{1}{1}{1}_1]{2}{2}_1{1}_1 \tag*{three one}
		\]
		\[
			L_1 = 312211
		\]
		\clearpage

	\section{The Function}
	Creating a function, $L(x)$, that generates a term, $L_{x + 1}$, given a previous term, $x$, 
	would first involve creating the comparison function, $\Delta(x)$, which produces a 
	corresponding $\Delta_x$. Then, by iterating through each pairs of $1$s within $\Delta_x$ and 
	appending each digit of $L_x$ and its ``count'' together, the next term, $L_{x+1}$, can be 
	produced.

		\subsection{Writing $\Delta(x)$}
		As previously explained, $\Delta(x)$ consists of appending comparisons for each pair of 
		digits within $x$ and padding them with $1$s on the ends. Given that $q(x, y)$ is the 
		equality function$^1$, and $d(x, i)$ is the digit-at function$^2$, a comparison at digit $i$
		can be done as follows:

		\begin{center}
		$comp(x, i)$ = $q(d(x, i), d(x, i + 1))$
		\end{center}

		\noindent If digit $i$ and digit $i + 1$ in $x$ are equal, this comparison results in a $1$, 
		otherwise, it evaluates to $0$. This is the reverse of what was described earlier, as the 
		differences between the digits should result in $1$s, while the equalities should result 
		in $0$s. Taking this into account:

		\begin{center}
		$comp(x, i)$ = $1 - q(d(x, i), d(x, i + 1))$
		\end{center}

		\noindent Now the comparisons are consistent with what was described earlier. In order to 
		append these comparisons, it's important to realize that any number in base 10 can be 
		written as the sum of its digits each multiplied by $10$ raised to that digit's index. For 
		example:

		\begin{center}
		$15012 = 1 \cdot 10^{4} + 5 \cdot 10^{3} + 0 \cdot 10^{2} + 1 \cdot 10^{1} + 2 \cdot 10^{0}$
		\end{center}

		\noindent Applying this to the comparisons, and given that $n(x)$ is the number-of-digits 
		function$^3$, the unpadded $\Delta_{x}$ can be calculated as follows:

		\begin{center}
		$\sum_{i = 0}^{n(x)} {comp(x, i) \cdot 10^i}$
		\end{center}
 
		% Creating $\Delta(x)$ consists of iterating through each pair of consecutive digits in $x$,
		% and ``comparing'' them. Appending these ``comparisons'' in order of evaluation will yield 
		% $\Delta_x$.	Each comparison consists of grabbing a digit of $x$ at index $i$ and comparing it
		% with the digit at $i + 1$.

		% \begin{center}
		% $x = 1112\underset{1}{2}\underset{0}{1}$
		% \end{center}

		% \noindent Since, digits $0$ and $1$ of $x$ differ, the first digit of $\Delta_x$ must be 
		% $1$ (this is without regards to the ``padding'' of $1$s). Given that $q(x, y)$ is the
		% equality function$^1$, and $d(x, i)$ is the digit-at function$^2$ Any given ``comparison'' 
		% for digit $i$ can be described by

		% \begin{center}
		% $q(d(x, i), d(x, i + 1))$
		% \end{center} 

		% \noindent Disregarding the ``padding'', for any given ``comparison'', the right-most index 
		% at which the comparison occurs on $x$ will be the same index of the result of the 
		% comparison on $\Delta_x$. For example, for the first comparison,

		% \begin{center}
		% $
		% 	L(x) = \sum_{i = 0}^{n_10(\Delta(x)) - 2}{d(\Delta(x), i)\cdot((R(J(x), i) - i)\cdot{10} ^ {1 + 2c(J(x), i)} + d(x, i)\cdot{10} ^ {2c(J(x), i)})}
		% $
		% \end{center}

\end{document}
